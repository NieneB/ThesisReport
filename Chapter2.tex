\chapter[Background theory and concepts]{Background theory and concepts}
In sections \ref{pedestrians} and \ref{walking} concepts used in this research will be clarified to have unambiguous use of terms. Section \ref{puccini} introduces the Puccini method which is used for the design of the public space in Amsterdam. 
This chapter will be finished with summaries of related literature which helped forming this research and provide reference and support for the methods used.

\section[Pedestrians]{Pedestrians}\label{pedestrians}
\begin{description}\item[A pedestrian] is a person, by foot, with or without helping aids, that moves in the public space. The person is not a driver but a road user. Also persons with a wheelchair, rollator, roller-blades, skateboard or children's bicycle, as well as persons walking with a bike, scooter or motorbike in the hand are pedestrian according to the Wegenverkeerswet voetganger.~\cite{Crow2014}
\item[Mobility] is the freedom to choose to travel and sojourn in the public space, being able to make the trip, regardless its distance. Pedestrian mobility differs from other modes of transport, for it almost part of all other trips.~\cite{Sauter2010}
\item[Sojourning] is an important indicator for quality of public space, it includes activities as; recreational walking, waiting and residing in the public space.~\cite{Sauter2010} \end{description}

The Netherlands is climatic and geomorphological very favourable for walking.~\cite{Sauter2010} In cities almost all streets have two sided side-walks.~\cite{Sauter2010} Though, cycling is perceived much more important in the Netherlands as seen in the pecking order in traffic: public transport, car traffic, scooter, bicycle, pedestrian. In some regions of the Netherlands the soil is rather soft and peaty, in these circumstances the pavement has to be renewed every few years, to prevent sinking.
	
\section{Walking behaviour}\label{walking}
The rate of walking for a person is determined by many factors, environmental influences, personal influences or behaviour and biological influences. Here we focus on the environmental influences, in specific, walkability of the environment. Walkability is one of the many important determinants that influence the walking behaviour of elderly~\cite{Vine2012} therefore we try to sketch an overall overview of the walking outdoor determinants. We will base this on the Intervention Mapping (IM) protocol for systematic theory-and evidence-based planning for behaviour change developed by Bartolomew.~\cite{Bartholmew2011} It also helps to show where walkability is positioned, next to terms like accessibility and mobility. Figure~\ref{behaviour} presents the personal and environmental determinants that influence the outcome for walking behaviour of elderly. 

The blue square shows walkability as part of the environmental determinants. Our desired behaviour outcome is to increase the choice of walking, consequently increasing health and quality of life. Terms used in dedicated studies are accessibility, walkability, safety and activity space.~\cite{Vine2012} Accessibility is the ability to access needed facilities and enable persons with disabilities to gain access through for example elevators, audio signals, walkway contours etc. As stated in the previous section, mobility is the freedom to travel in the public space. Here we assume that all elderly have the freedom to go outside, make use of the public space and access the facilities they need. The quality of these routes, the friendliness and the safety are gathered in the walkability determinant. So while a person in walking and accessing the public space, how user friendly and how much effort does it cost to cover these routes?

Some important personal determinants are personal health, previous experiences and perceived self-efficiency. Personal determinants are are often influenced by the environmental determinants, the blue dotted line, but are not part of this study.

\begin{figure}[h]
\includegraphics[width=\textwidth]{img/B_bartolomew.pdf}
\centering
\caption[Walking Behaviour Scheme]{Walking Behaviour Scheme (Source: Boeijen, based on Bartholomew 2011)
\label{behaviour}}
\end{figure}

The individual path selection criteria of Golledge can be placed next to figure \ref{behaviour} to show how they influence the actual choice for walking and that the combination of personal and environmental determinants together make an individual decide to go by foot or not. When choosing a route and having experiences on these routes influence the choice for the next time the user wants to walk that route and which path will be followed.~\cite{Golledge2002} The individual path selection criteria by Golledge knows 3 phases~\cite{Golledge2002}: 
\begin{enumerate}
\item Choosing destinations 
\item Making a route 
\item Implementing and feedback. Confirmation or change?
\end{enumerate}
The first two phases are influenced by the many determinants, while the last phase provides new knowledge and experience again influencing the next time a user wants to walk outside. Increasing or decreasing the influence of the previous determinants. 


\section{Puccini Method}\label{puccini}
Amsterdam is a cultural historic city, with many urban designs from different time periods. The historical city centre, the 19th century canal-belt, the 20-40ties en the postwar city. The urban design has to suit the cultural historic character of the city. For every urban style zones there are material lists and standard design details.~\cite{puccini2014} A map of Amsterdam indicating where each urban time zone is located can be found in annex \ref{pucciniMap}.

The Puccini Method is the Amsterdam tradition for the design of the public space in Amsterdam formulating design principles. Public space includes all non-build-up space, open to and accessible by people. The Puccini Method contains all points of policy, from the detailed design to technical details and material lists. It contains four modules, red for streets, green for vegetation, blue for water and water banks, and last, purple for street furniture, street lights and public transport stops. 
The Puccini method is the handbook for the municipality to maintain and design the public space, it is not an obligatory policy design book.~\cite{puccini2014}

The Puccini method contains 6 convictions for the design of the public space :~\cite{puccini2014}
\begin{enumerate}
\item 	\begin{description}
		\item[Choose, not share] 
		The streets are used more intensive and pressure increases. Often the usage pressure is higher then the available physical area. Therefore the available space has to be assigned to one use, not shared. 
		\end{description}
\item \begin{description}
		\item[Simplicity and obviousness] 
		The pbulic space should be user-friendly, sustainable, strong in simplicity, timeless and obvious. With simple material and simple design, reaching for good quality.
		\end{description}
\item \begin{description}
		\item[Craft and skill as a basis] Craft expertise is the basis. Not only designing inside, but going out on the streets with work experience is the key. 
		\end{description}
\item \begin{description}
		\item[Crucial eye for detail] Detail for material use, time, financial resources and facing problems that have to be solved. 
		\end{description}
\item \begin{description}
		\item[A good plan is maintainable] After the realization of a street or square it still has to be maintained. The plan has to take the maintenance into account for the future. 
		\end{description}
\item \begin{description}
		\item[Cooperation] A lot of specialised disciplines are concerned. Throughout the whole process it is important to communicate until the end product is realized. 
		\end{description}
\end{enumerate}

\section{Walkability factors from literature}\label{Bliterature}
The rollator is a Swedish invention. The Swedish researcher St\"al focusses a lot on research of elderly pedestrians with rollators, on their accessibility and their safety in the public space and how interventions in the public space impact the elderly pedestrian. St\"ahl is one of the few that looks specifically at the elders perception of the built environment and how specific interventions can help rollator walks be perceived more attractive.~\cite{Stahl2008, Stahl2013} An important part in her research, is user involvement, which leads to research with greater relevance and improvements to outdoor pedestrian environments based on the problems identified by elderly people. St\"al claims to be among the very few examples of user involvement in research targeting at societal panning. A mixed-method approach is used emphasising involvement of elderly people, not only as sources of data but also as partners.~\cite{Stahl2008} Conclusions from the studies are that different individual background variables influence older peoples' perceptions of the walkability factors. Perceptions on the outdoor environment are influenced by sexes, functional limitations, use of mobility devices and age. For example, physical barriers in the outdoor environment become increasingly evident with increasing age, among older people with functional limitations and users of mobility devices.~\cite{Stahl2013, Wennberg2009}

% The measurements most mentioned are:
% \begin{enumerate}
% \item Separation of pedestrians/cyclists
% \item Lower speed limits
% \item Better maintenance
% \item Wider side walks
% \item Decrease curb levels
% \item More even surfaces on pavements
% \end{enumerate}~\cite{Stahl2008}


Several other studies, not necessarily aiming at elderly with a rollator or user involvement, made an overview of walkability criteria for pedestrians, elderly or the mobile impaired. Verschuur~\cite{Verschuur2013} provides a list of parameters affecting route attractiveness and the studies in which its found, for elderly pedestrians. Duncan (2011)~\cite{Duncan2011} provides a whole list of walkability indicators used for calculating a WalkScore for the general pedestrian. Rosenberg~\cite{Rosenberg2012} provides a summary of barriers and facilitators in the built environment for mid-life and older adults with mobility disabilities. Wennberg studies a whole list of usability factors and their statistical significance, divided into categories, physical barriers, orientation and warnings, bus stops and shops, orderliness and benches and stairs, for elderly pedestrians~\cite{Wennberg2009}. 

All these lists and other studies are used to create an own overview in order to filter out the most important walkability criteria for elderly with a rollator~\cite{ Bernhoft2008, Verschuur2013, Dunbar2004, Wennberg2010, Borst2008, Rosenberg2012, Vine2012, Matthews2003, Hovbrandt2007, WWT2012, Wennberg2009, Stahl2008, Stahl2013}. The total list can be found in Annex \ref{Acriteria}.Words used for searching are pedestrianism, elderly pedestrians, mobile impaired pedestrians, rollator users, walkability studies and walking behaviour. The focus is on the criteria that are influenced by a wide variety of factors in the urban environment. In order to analyse which level in the urban environment is the most important to focus on, all the criteria are made more orderly, by sub-diving them into tree levels of where they could occur. 

\begin{enumerate}
\item Pavement level
\item Street level
\item Environmental level
\end{enumerate}
Additional, weather related and temporal issues can occur over all these tree levels. Also crossings form a own special niche in the levels for it is not part of the pavement or the street. Therefore, these levels are added to the three levels mentioned above.
\begin{enumerate}
\item Weather or temporal level
\item Crossings
\end{enumerate}

Next to this, the factors are assigned to different categories focussing on their characteristics such as accessibility or quality of the environment. Five categories can be distinguished. These are inspired by~\cite{Ballester2011} and~\cite{Rosenberg2012}.

\begin{enumerate}
\item Accessibility
\item Quality 
\item Obstructions or barriers
\item Route attractiveness 
\item The feeling of safety
\end{enumerate}

The category accessibility describes the access and presence of pavements. Some examples of factors are: the availability of public transport nodes, availability of bridges, availability of ramps to the pavement a.o. 
The quality is about when a pavement or stair is present, what the quality is. This could include things as a non slippery pavements, the slope of bridges and the slope of the pavement. 
Obstruction contains tangible objects that are in the way, temporary or stationary. Like protruding portals and facades, blocking commercial signs or green maintenances so branches are not hanging over the pavement.
Route attractiveness is more about the feeling one has towards the routes, the attractiveness can go down with the presence of dog droppings on the pavement, litter and garbage on the streets. While it can go up with the availability of resting benches, public toilets, green and trees along the route. 
The feeling of safety includes factors as enough time to cross the street, good overview while crossing a street, vehicle-pedestrian interaction, speed limits, presence of street lights, the amount of criminality and many others. 

\section{Related literature for methodology}\label{literature}
Literature that discusses the possible methodologies to make walkability quantifiable are rare. Matthews et al. (2011) developed a model for mapping accessibility for wheelchair users and comes closest to this study. Svensson (2010), researched accessibility in particular. Finally, Walkscore from Duncan et al.(2011) is a project focussing on mapping neighbourhood walkability to enhance physical activity among children, adolescents and adults. It provides an interesting method for mapping several walkability indicators. 
The term Smart Walker is used when referring to measurements with sensors on a rollator in order to increase the capabilities of the rollator. The studies from Wang et al.(2015) and Weiss et al.(2014) focus on the technical aspect of measuring rollator movements and gait characteristics with an accelerometer. The next sections, will give a more detailed summary of the above named articles. 

\subsection{Mapping walkability}
A wheeled walker and wheelchairs have a lot in common. Therefore, also literature studies aimed specific at wheelchairs can provide useful insights for rollator research. Matthews et al. 2003 uses a GIS based system to show that the built environment is often distorted and forms a hostile space for wheelchair users~\cite{Matthews2003}. MAGUS, modelling Access with GIS in Urban Systems, is a multi-criteria assessment model using quantitative and qualitative techniques to provide urban planners and disabled people with up-to-date, detailed and customized information. This can help the elderly to plan and manage access and mobility in urban areas. At the same time, this provides planners a system that help them evaluate the effects of their design decisions on the mobile disabled pedestrians.~\cite{Matthews2003} The model is based on real-world perceptions, experiences and needs of disabled people. The eventual choices included in the model resulted in all possible routes, avoiding bad surfaces, avoiding slopes with a gradient of more then 4\% and using only controlled crossings. A drawback in the data was that no information is held on pavement centrelines which forms the starting layer within a GIS model. The solution was manual plotting, and taking into account building outlines, walls, road edges and other linear and point details. So a precise pedestrian route can be mapped. The MAGUS project has been installed at various places in the UK, the report states new fund were available for the MAGUS model to become web-based.~\cite{Matthews2003} This however was in 2001. Recent developments on the MAGUS project are unknown. 

Svensson (2010)~\cite{Svensson2010} uses a GIS-model to map and measure accessibility of the urban environment for citizens whit impairments. In order to improve accessibility planners require knowledge about the location of obstacles and how these affect accessibility. Data on the physical environment of different Swedish towns has been collected in order to identify neighbourhood specific characteristics. A digital model of the town's pedestrian network contains information about: slope, height of kerb stones, type of pavement and width of a side walk. They found that only a fraction of the population is able to reach a bus stop from their home, if they are allowed to use only those parts of the pedestrian network that are defined as usable, for mobile and vision impaired citizens. Also the study shows that accessibility varies with different neighbourhood types~\cite{Svensson2010}. Overall the study demonstrates that a GIS system could be used to gain vital information about the need for improvement in the built environment. Rather easily, the knowledge about the urban environment enhances how certain obstacles and flaws affect accessibility for the impaired citizens.~\cite{Svensson2010}

WalkScore presents an web-based geospatial technology to estimate neighbourhood walkability in an interesting way, a weighted multi-criteria method, which could be used as inspiration to extent such methods to a more detailed survey for the mobile impaired pedestrian. It is developed for assessing neighbourhood walkability by Duncan et al.(2011)~\cite{Duncan2011}. The WalkScore algorithm is based on GIS indicators of neighbourhood walkability and ranges between a score of 0-100. It is developed for a large area in four US metropolitan areas with several street network buffer distances, 400, 800 and 1600$m$ buffers. Duncan et al. does not focus on the mobile impaired but at improving physical activity among children, adolescents and adults. 

\subsection{The Smart Walker}
There are a few researches on making a rollator more smart by using sensors; the Smart Walker. Most of these studies focus on fall protection, early warning systems or indoor navigation. There are functions such as sensor assistance, health monitoring, navigation help or cognitive assistance, obstacle avoidance and fall detection.~\cite{Wang2015} Most studies using an accelerometer as a sensor, calculate gait parameters such as; step length, gait cycle, step width and gait variability.~\cite{Wang2015} Often these studies are done in laboratory settings, using high quality sensors and sensors attached to the body. Our own research aims at obstacle detection, with the help from detecting the movement of the rollator. At this moment in time, conducting this research, no research of this kind had been published.

Weiss et al. (2014) uses a smart walker to detect walking behaviour change in real time with an accelerometer mounted on the rollator.~\cite{Weiss2014} No sensor on the user is needed, but the walking behaviour is detected based on the motion transfer by the user on the walker. The goal is to identify five different classes: no movement, movement, slow, normal and fast. They confirm that walking behaviour changes can be detected by using a 3-axis accelerometer sensor on the walker.~\cite{Weiss2014}

\begin{figure}[h]
\includegraphics[width=0.5\textwidth]{img/B_Weiss.png}
\centering
\caption[Rollator with accelerometer axis configuration]{Rollator with accelerometer axis configuration. From Weiss et al.~\cite{Weiss2014} \label{rolaxis}}
\end{figure}
\begin{figure}[h]
\includegraphics[width=0.5\textwidth]{img/B_Wang.jpg}
\centering
\caption[Rollator set-up]{Rollator research set-up. From Wang et al.~\cite{Wang2015} \label{rolsetup}}
\end{figure}

Wang et al. (2015) uses a standard 4-wheel rollator with a defined walking frame, where the accelerometer is positioned in the middle point between the two rear wheels. By doing this the exact distances to the wheels are know and from this the position change of the walker can be determined. By using a high cost motion capture system, the calculated trajectory's and the gait or step detection, are validated. Wang was able to calculate the displacement of the walker during every step with an accuracy of approx 1 cm.~\cite{Wang2015} By comparing a group of young adults with group of elderly people, Wang found the the walking accuracy of the elderly is lower but that step length, step period and walking speed between the two groups has no obvious difference. The classical gait indicators are not sufficient or sensitive enough to evaluate the fall risk of elderly.~\cite{Wang2015}

\subsection{Concluding}
The MAGUS model of Matthews and the Walkscore from Duncan, both use a multi-criteria model. MAGUS including criteria as avoiding slopes, cobbled pavement surfaces and using controlled crossings as input. Walkscore, was more general, and focussed more on accessibility, using network length and network nodes density. Svensson focussed more on comparing different neighbourhood designs and collected data on slope, height of kerb stones, type of pavement and width of a side walk. Both Wang and Weiss made use of a non-intrusive measurement method, were no devices where placed on the participants but only on the rollator. In order to measure gait characteristics and walking behaviour. 

For this study both methodologies will be combined. First, the walkability factors are mapped using existing geo-data. Second, the concept of a Smart Walker with an accelerometer will be used to map more in detail the effects of obstacles in the build environment on the walking behaviour and movement characteristics of the rollator. 