\chapter[Discussion and conclusions]{Discussion and conclusions} 
%  restate the research question. nr 1. Critical walkability factors
The critical walkability factors for urban outdoor environment for elderly people, with a rollator, show that there are many things mentionable to improve on, and policy does not take pedestrians into account sufficiently. When policy design documents and the contacts from the municipality state that the pedestrian area is often the residual area in design and has the least priority, the arguments from the introduction are confirmed, that indeed, the pedestrian is forgotten in design. Moreover, till so far this applies for the total group of pedestrians, not, the mobility impaired pedestrians, who more require a well designed public space. 

The small part of interviews held with rollator users, underline the importance of the quality of the pedestrian area. When surfaces will be more smooth, and surroundings less exhausting, longer and more comfortable walks will be able. Keeping the elderly longer fit, enabled and so living longer independent. Though, personal behaviour and preferences strongly influence the feelings and emotions one has to its surroundings. Shown in the two interviews, where one participant really likes walking while the other rather stays inside.

From the final list of critical factors assembled by all the findings, one of the main irritations specifically for Amsterdam are wrongly parked bikes and cars. Mainly due to the small pavement areas and parking on the pavements. Many road users, hardly think about the less impaired road users while parking their bike outside, against a lantern pole, and so blocking the pavements for wheelchair or rollator users. 


%  restate the research question. nr 2. Mapping the Critical walkability factors
% The most mentioned critical walkability indicators which are researchable with GIS methods are sloping pavements

% The critical factors derived from the available geo-data is sloping surface and the amount of area available for pedestrians. 

The GBKA showed that also in the available geo data on the public space, the pedestrians are forgotten, as no label was attached with which target group the dedicated polygon belongs to. Also no information could be found about the exact type of material of the surface or maintenance activities. Only an overview map showing material use per neighbourhood is available, see the Puccini map in annex \ref{pucciniMap} 
Besides, there is no indication whatsoever on where curbs are located, how high they are, and whether there are ramps placed or where the curb is lowered. 
This made it all hard to derive a walkability overview with the available geo data and create a map with possible bottlenecks. A own approach to determine which areas are pedestrian area, showed rather high numbers for pedestrians. Around 53\% of the public space in the test area Jordaan, belonged to pedestrians. This did include parking plots and public squares. A surprising amount as often in Amsterdam, pedestrians pavements are very narrow. Though this area did not take into account the placement of poles, bike racks, traffic signs or any other object placed on the pavement. Better was not to include parking plots into the pedestrian area. 

When looking at sloping pavements, one of the other criteria found by research objective 1, the allowable 4\% of slope for pedestrian pavement reduced the optimal pedestrian area to only 8\%. Showing a lot of area being too steep for pedestrian allowable limits. While roads, for motorized transportation have less steep slopes, around 38\% is below the 4\% slope. 




%  restate the research question. nr 3. Mapping the Critical walkability factors
Map and analysing the critical factors, by measuring rollator movements whit an accelerometer, showed that irregular surfaces can be measured with a accelerometer. The different kind of surfaces showed clear distinctions in the amount of vibrations when looking at the variance over the measured track. With grass and stones having the highest variance and vibrations and smooth concrete the lowest. When comparing this to a research done by Matthews et al. (2003) a correlation of 0.72 is found. Indicating the two studies do correlate but not highly. This, because the surfaces of Matthews and this research were probably not the exact same, what was measured differs and measurement methods differ. Matthews measured surface hindrance as the accelerometer measures the vibration effect on the rollator. Though, the more vibration the more hindrance exists. That the same kind of material show approximately the same relative hindrance values show that surface hindrance can be approach with an accelerometer as well. Possible applications might be to detect the kind of surface material from the accelerometer signatures. 

%  restate the research question. nr 2. Mapping the Critical walkability factors
Combining the two methods, existing geo data and measurements with an accelerometer, showed new possibilities in detecting change points in the walking routes. 




%2. Relate your findings to the issues you raised in the introduction. Note similarities, differences, common or different trends.  Show how your study either corraborates, extends, refines, or conflicts with previous findings.

% 3. If you have unexpected findings, try to interpret them in terms of method, interpretation, even a restructured hypothesis; in extreme cases, you may have to rewrite your introduction. Be honest about the limitations of your study.

%4. State the major conclusions from your study and present the theoretical and practical implications of your study.


% 5. Discuss the implications of your study for future research and be specific about the next logical steps for future researchers.
Possible  applications of smart phone with accelerometer use, for measuring rollator movements or surface resistance. Translate this to the amount of energy that elderly consume. 









For elderly, who are more vulnerable, environmental attributes can be barriers to an active engagement in urban life. The quality of the immediate environment is a significant determinant of elders well-being, independence and quality of life.

- non intrusive, low cost method. 
- no difference in walking gait characteristics between elderly and young people.~\cite{Wang2015} So not difference from walking with the rollator myself. 




Now, the urban environment will never be optimal, so certain change points will indicate not so important happenings. 



Safety, robbery and perception of safety are hard to classify  (hang jeugd)
Perception of safety is different for every individual. Very personal determinants. 
 

Elderly:
Lot of differences in rollator users
Lot of different rollators, this research does not take into account the type of rollator. But the type of wheels and structure of frame might really influence the measurements. 