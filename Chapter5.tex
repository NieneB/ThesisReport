\chapter{Conclusion, discussion and recommendation} 
The first section of this final chapter restates the research questions and presents the main conclusions of this research. Section 5.2 discusses the outcomes and limitations of the research. Section 5.3 gives several recommendations to improve this research and for possible future research.

\section{Conclusion} %antwoord op vragen geven
The objective of this research is to analyse and visualize geodata to explore the critical walkability factors for elderly dependent on a rollator in the urban outdoor space. With the goal to raise more awareness and shed more light on the forgotten elderly pedestrian. To achieve this objective, the following research questions were answered:

\begin{enumerate}
	\item The first research question found the critical walkability factors hindering elderly with a rollator in the outdoor environment.
	\item These critical walkability factors were mapped and analysed with existing geodata in research question 2. 
 	\item Research question 3 collected own geodata with a smart walker to map and analyse these critical walkability factors.
	\item In the end both the available and the measured geodata are compared in research question 4. 
\end{enumerate}

% restate the research question. nr 1. Critical walkability factors
The most mentioned critical walkability factors found in literature study and interviews were, wrongly parked bikes and cars, irregular cobbled surfaces and sloping pavements. There are many criteria mentionable to improve walkability for the elderly pedestrian. User experiences and own experiences of this research underline the importance of the quality of the pedestrian area. The most important thing mentioned by rollator users is with smoother surfaces and less exhausting surroundings, longer and more comfortable walks can be made. This research found walkability factors in line with previous researches and contributes to the knowledge specifically for Amsterdam. One of the main irritations specifically for Amsterdam are wrongly parked bikes and cars, mainly due to the small pavement areas and parking lots on the pavements. We noticed that many road users hardly think about the less impaired road users while parking their bike outside against a lantern pole and thus block the pavements for wheelchair or rollator users.

The research to policy design perspective and interviews with the municipality showed that policy design does not take pedestrians into account sufficiently, let alone the elderly or mobile impaired pedestrian. There was no data available on elderly pedestrians and no data collected at the decentralized governmental level, which is responsible for the design of the public space. Currently the people that we corresponded with, working on public space design, are not aware of any data on pedestrians or where to possibly find it. Guidelines about the design of pavements were found in a report from 2011.~\cite{leidraad2011} If these guidelines are followed in Amsterdam is not sure. The study by Verschuur (2013) developed a method to test the implementation of these guidelines in a case study for Utrecht, the Netherlands. In order to do this, a basic geodata layer is needed showing the pedestrian area. 

The works of Matthews et al. (2003), Modelling Access with GIS in Urban Systems~\cite{Matthews2003}, Svensson (2010)~\cite{Svensson2010} and Duncan et al. (2011) with Walk Score: Estimating Neighbourhood Walkability~\cite{Duncan2011} provide methods for quantifying walkability factors in a GIS system. They use a multi-criteria model with quantitative and qualitative techniques to make visible how the build environment can be hostile for the mobile impaired and need a basic geodata layer about the area used for pedestrians. When looking at available geodata in Amsterdam for mapping walkability, there was no dataset at the Amsterdam municipality on the location of curbs, their height, where ramps are placed or where the curb is lowered. Also they could not provide information about the exact type of material of the surface or maintenance activities. We did collect one overview map showing material use on neighbourhood level, see the Puccini map in annex \ref{pucciniMap}. The only data available was the GBKA showing a detailed topology map of the centre of Amsterdam. Labels indicating pedestrian area are missing. To see if this could be derived from the available GBKA an own approach is developed to determine which areas are pedestrian area. It showed high numbers for pedestrians: around 53\% of the public space in the test area Jordaan, belonged to pedestrians. This is a surprising high amount because often in Amsterdam pedestrians pavements are very narrow. The assumptions made in the classification, included parking lots and public squares within the pedestrian category. Also, pedestrian area often contain a lot of poles, bike racks, traffic signs or any other object placed on the pavement. The GBKA showed that the available geo-data of the municipality still lacks detail and information and is not suitable for quick pedestrian area analysis. This confirms the statements of the introduction that pedestrians are missing in data.

Mapping the slope with the AHN2 to derive sloping pavements above 4\% showed that a lot of pavements classified as pedestrian area had an average slope higher than 4\%. While road segments classified for motorized transport showed average values below the 4\% slope. The 4\% slope rule, reduced the optimal pedestrian area to only 8\% of the total surface, including parking lots and public squares. This shows that a lot of area is too steep for pedestrian allowable limits and can be labelled as unsuitable. In conclusion, the walkability criteria of sloping pavements can easily be mapped with the AHN2.

Mapping the walkability criteria with our own smart walker was done by measuring location and rollator vibrations with a GNSS and Accelerometer. The works from Weiss et al. 2014 and Wang et al. 2015~\cite{Weiss2014, Wang2015} gave the inspiration to work with the concept of a Smart Walker. Several test routes were walked during the rollator loop but failed to measure the accelerometer sensor and were therefore not usable. A second test did succeed and the rollator measuring different surfaces showed that irregular surfaces can be measured with an accelerometer. The different kind of surfaces showed clear distinctions in the amount of vibrations when looking at the variance over the measured track. Grass and stones have the highest variance and vibrations and smooth concrete the lowest. When comparing our results to a similar research done by Matthews et al. (2003) a moderate correlation of 0.75 is found.

In order to compare the existing geodata to the measured geodata a changepoint analysis was used. The changepoint method shows an interesting concept for detecting obstacles during a walking route. Several locations can be indicated where the changepoints in the walking behaviour (speed), the slope of the location, and the accelerometer behaviour fall together and could indicate a large obstacle like a curb. See detailed map (c) in figure \ref{routeM} and detailed map (b) and (c) in figure \ref{routeLeicaMap}. Peaks in the acceleration of the z-axis are located around imperfections in the pavement and are not accompanied by speed changes, indicating a small obstacle, see map (b) in figure \ref{routeM} and (c) in figure\ref{routeB2}. Overall it is interesting to notice that almost all changepoints can be explained through logical reasoning. In the speed time series of the Leica route all changepoints indicate a obstacle, curb, a break made by the rollator user or curves in the road. Also the changepoints in the z-axis acceleration in the walking route with the MeetRollator seem to be related to the true situation. However, poor GPS performance makes it hard to accurately compare the points location. Overall, on straight segments of the route where the pavement stays nearly the same and walking speed is continuous, less changepoints can be found. A high amount of incorrect slope changepoints originates from the previously mentioned inaccurate location determination which resulted in extracting values from the slope raster at the wrong location. 

That the accelerometer did result in interesting changepoints at curbs and obstacles and clearly detected differences in surface characteristics shows that obstacle detection and surface quality monitoring is another possible application to the Smart Walker. By using the presented methodology more data can be gathered on the critical walkability factors for elderly with a rollator. It provides a method to easily map obstacles and pavement quality that can help for design reference evaluation and inventory. The goal of this research is to support the forgotten pedestrian in policy, research and data. By using these methods to map the walkability problems we raise more awareness and hope to trigger possible action at decentralized governments to increase walkability quality and so increase walking activity of elderly people. To keep them fit and healthy, capable to live independently and grow old in their own house and environment. All contributing to less need for healthcare to the growing share of elder population and reduce the cost of elder care.

\section{Discussion} %bevestiging tov literatuur
We concluded that the accelerometer together with GPS attached to a rollator provides an interesting approach to measuring surface irregularities and locating obstacles during a walking route. Also the AHN2 proved suitable for mapping the sloping pavements. However, the process of this research and the methods used may have many pitfalls, influencing the accuracy of the outcome. In this section we discuss the methods and the results. 

\subsection{The forgotten pedestrian}
Policy design documents state that the pedestrian area is often the residual area in design with the least priority. This was confirmed by the contacts from the municipality giving arguments that the pedestrian is forgotten in design. This refers to the total group of pedestrians, not specifically the mobility impaired pedestrians, who even require a better designed public space. 
Our contact persons at the municipality could not tell if any geodata was available on pedestrians or who could provide it. This could mean several things: first of all, the data does exist and the people we spoke to, working on the design of the public space, are not aware of possible data. Confirming that the pedestrian is forgotten in policy design and data is scattered and not well organized. Second, the data does exist but we had the wrong contact persons at the municipality. Final option: the data does really not exist. This confirms the forgotten pedestrian in data like the PQN report states that there is not enough data on pedestrians~\cite{Sauter2010}. 

Matthews et al. (2003)~\cite{Matthews2003} also encounters this problem, in specific that no data is held on pavement centrelines. The basic layer for a GIS model is to know what is pedestrian area or not. Our own research confirmed that the GBKA, the most detailed topology dataset of the municipality of Amsterdam, did not contain a label on roads to determine whether it was a road for cars or a pavement for pedestrians. In order to find the pedestrian in data we conducted our own analysis of general assumptions to label the pedestrian polygons. This task proved itself more difficult than expected. The developed approach showed a lot of wrongly classified segments because many assumptions caused wrong classification. For example, not all parking lots are suitable for walking. Bridges often have pavement on either side, but are not detected in the approach for they rarely contain street furniture and are not adjacent to buildings. Matthews suggest to map precise pedestrian routes manually, but this was not done for this study as it was too time consuming~\cite{Matthews2003}.

\subsection{Collecting the user perception}
As Stahl et al. 2008 stated, user involvement leads to research of grater relevance to people and the findings more likely to be implemented~\cite{Stahl2008}. Their research is based on the problems identified by elderly people and therefore of great relevance. Also for this study the involvement of elderly with a rollator was perceived as important. For the conducting researcher to be young and not having any experiences with mobility problems, it was key to hear from first hand. The first intention of our research was to interview at least 20 elderly in Amsterdam that walk outside regularly with a rollator. Unfortunately, this was harder than expected. Elderly care houses were not keen on cooperating. Calling would often result in a redirection to the location manager and no answer to the e-mails. Some institutions did show some enthusiasm: The Flessenman and De Tweede Uitleg, were willing to cooperate but could only provide 3 participants who walked outside with the rollator from which 2 were available for interviewing. The Buurtzorg Centrum were also enthusiastic. But after several calls and e-mails they never responded with possible contacts for participants. In the end the short and quick interviews at the Rollatorloop gave more insight. The participants that did cooperate showed how strong the influence of the personal preferences is on the perception to the surroundings. One of the participant really likes walking while the other rather stayed inside. This strongly influenced their feelings and emotions they had to walking outside. The presence of differences in rollator users and personal determinants should also be taken into account for walkability research. 

Another interesting question is why elderly people in the care house did not go outside? The Flessenman counts 183 rooms~\cite{flessenman} and the caregivers only responded with 3 participants that walked outside regularly, which is a small number. This however was not asked for or looked into during this research. 

\subsection{The Accelerometer sensor of a Smart phone}
Next to different rollator users there are also a lot of different rollators. This research did not take into account the many types of rollators available. The type of wheels and structure of frame might influence the measurements and can result in different data characteristics if multiple rollators are used.

The application, Physics Toolbox Accelerometer, showed good results and was easy to use for measuring the accelerometer sensor of the phone. However, the first versions of the application stopped measuring when the phone went in sleeping mode. The test set-up showed imperfections caused by a lack of experience of the conducting researcher, who was not familiar with these kind of datasets. The test data could have been explored better in the first stage for the missing data gaps, when the phone went in sleeping mode, could have been detected earlier. After a quick analysis assumed was that the sensor had a malfunction for example, heavy movements cause the sensor to overload. After the first experiments failed, a more in depth research of the application behaviour and its settings, including contacting the developer, was done. This resulted in more controlled knowledge about the performance of the application. After the application version change which improved the application, it was perfectly working and gave no further problems. 

The exact accuracy or errors from the accelerometer are unknown, as the quality of the sensor in the phone is unknown and not researched. The exact meaning of the value range given by the application are unknown and can differ per phone device.\footnote{Information provided by Chrystian Vieyra, developer of the Physics Toolbox Applications. In e-mail conversation 26-10-15. } This could be relevant if our data has to be compared to other measurements and methodologies. Within our own study, the values are all relative to each other and can be compared. Also, observation of the measurements showed credible outcomes because the derived changepoints showed a realistic situation. This was confirmed by putting the accelerometer flat on the table which did not show any abnormal peaks or deviations. 

The comparison with Matthews et al.~\cite{Matthews2003} showed a moderate correlation. The measurement methods are different and the exact type of surfaces used by Matthews unknown. A correlation for a linear increase in the variation per surface type can be seen in both data sets. Matthews measured surface hindrance and we measured the vibration effects on the rollator, both can be seen as a measure of the roughness of the surface. 

\subsection{Influenced experiments}
After the measurements with elderly walking with a rollator failed, we conducted some additional walks with the MeetRollator ourselves. This could give a wrong image on the measured walking behaviour outcome. From Wang et al.(2015) we learned that there is no difference in walking gait characteristics between elderly and young people~\cite{Wang2015}. But elderly are more used to handling a rollator compared to young people, for which a rollator could be an obstacle in moving more easily. Walking with the rollator myself could therefore have influenced the outcomes. 

\subsection{Spatial accuracy in data and measurements}
The locations measured with the smart phone or the Garmin summit are not accurate enough to detect the location on a few centimetres exact. This resulted in wrong extraction of data values from the AHN2 and the slope raster. Many slope changepoints cannot be used to base a conclusion on. Also the detected changepoints from this dataset which are mapped with the location measurements did therefore not represent the true location where they occur. 
 In order to detect exactly where a route was walked, a good location determination is needed. The detail required, is almost on stone level. The measurements are not useful for walkability studies if it cannot be determined if someone is walking on the right side of the street or the left. Also on pavement level the presence of put-holes or curbs is of significant importance. The Leica system can detect the location on a few centimetres accuracy. But it is a heavy system which cannot be easily mounted on a rollator without influencing the rollator control and movement. It could possibly cause a dangerous situation for the elderly depending on it. 

%inacuracy in the slope and AHN
The slope map resulted in classifying most pedestrian area as unsuitable, for it contained an average slope of above 4\%. This was influenced by the polygons of the pedestrian area which include the curbs. Also building edges not being cleaned out in detail enough can result in very high slopes in some pixels, resulting in an average slope which is way too high. The AHN2 dataset which was used is cleaned from parked cars and trees but obstacles or bikes parked on the pavement are not erased. This also, can cause an average slope that is too high. Overall, the AHN2 has a resolution of 0.5$m^2$ and a precision of systematic and stochastic error of max 5$cm$~\cite{VanDerZon2013}. Which makes it detailed enough for mapping pavement slope and curb detection.

Mapping the accelerometer points with the GPS was done in the RDnew projection and could therefore be easily approached with a linear relation. The accuracy of the GPS measurements is again the accuracy of mapping the changepoints. For this is the starting point where the time and distance difference is key. Here we assume the measured time of the accelerometer and the GPS are to the millisecond accurate. 

\subsection{The Changepoint method}
The changepoint detection methods as provided by Killick et al.~\cite{changepoint2015} was an unknown method to the conducted researcher and her supervisors. When first encountering the method it showed a lot of potential and interesting opportunities to analyse statistical route characteristics. 

First, the possible settings in the changepoint package of R had to be explored. Killick et al.~\cite{changepoint2015} provides several algorithms and penalty settings. The reliability of the changepoint detection depends on the algorithm choices and settings and greatly influence the accuracy. Because there are many different possibilities ours is maybe not the best. Next to this, missing data gaps were filled in with an average value, influencing the time series characteristics. The changepoint detection did provide a quick and easy method to find statistical changes in the time series data. The accuracy of the detected changepoints can only be checked if the location is also accurately determined. Now, we can only assume that the indicated events found are associated with obstacles like a curb. 

In accuracy in the measurements showing changepoints that do not locate a physical obstacle can for example be caused by the start of walking. A firm push against the rollator to start of, shows unimportant changepoints in the vertical acceleration. Also a break on the street will show as not so important changepoint in speed. The urban environment will never be optimal and some changepoints will indicate not so important happenings. 

\section{Recommendations} %hoe nu verder. Wat meer? LIJST
Recommendations to improve the methodology:
\begin{itemize}
\item Conduct a more structured interview methodology related to specific routes and critical factor analysis, with elderly living in Amsterdam and walking with a rollator outdoors, to get a better understanding of location specific problems. 

\item Conduct more controlled experiments with elderly themselves, walking routes with the rollator. Using a good GPS system and a good working accelerometer application. Also comparing routes walked in a controlled environment against routes walked in the real living environment. This can give better insights if this methods works for detecting obstacles along the real walking routes of elderly, when walked by themselves. Instead of a young vital person walking with a rollator. 

\item Use more precise GPS systems, like the Leica system for location determination. The detail for which the detection of obstacles is needed, has to be very precise. The slightest deviation can give falsehoods in the changepoint detection for slope. Another possibility is the method used in the research of Wang et al.~\cite{Wang2015}. Here they use an accelerometer exactly placed on the rollator to monitor the displacement for every step made. By using this method the route of a person could be exactly determined from the starting point, with an accuracy of about 1 cm and no GPS would be needed. This would require a standardized measurement rollator were the exact distances of the sensor to the wheels and frame is known.

\item In order to conduct a better pedestrian area classification, it would have been better not to include parking lots into the pedestrian area. Comparing the classification methods with manual classification could have said more about the accuracy of the method. 

\item More research to the application and possibilities of the AHN2 can be interesting for mapping walkability. The accuracy of the AHN2 is detailed enough to map the presence of curbs. It would be interesting to see if it can also map the height of curbs, another critical walkability factor that is of importance for elderly with a rollator.

\end{itemize}

Recommendations for future research:

\begin{itemize}
\item Quantify the relation between multiple determinants for walking behaviour as showed in figure \ref{behaviour}. The perception of elderly to different circumstances could be measured. An interesting link could be the report from Hogertz et al. 2010~\cite{Sauter2010}, which tries to measure skin conductivity as a quantitative measure for arousal. 

\item Transform the surface hindrance into a value of energy effort that is needed by elderly with a rollator to walk there. For example measuring the amount of energy elderly consume. Creating a sort of wattage of elderly people

\item Possible applications might be to detect the kind of surface material from the accelerometer signatures. Create a specific signature for a specific surface material. Different sets of rollators might react differently to surface irregularities and has to be taken into account.

\item This study could be used to create a low cost, non-intrusive method for identifying the location of obstacles on the route of elderly with a rollator. A more in detail method could be designed, that automatically filters out priority obstacles when the data changes in a specific way. Smart phones attached to rollators of elderly can provide automatic data on the surface hindrance and obstacles they face. 

\item A final but essential recommendation is develop a method to detect wrongly parked bikes and cars. OR set up a program to raise more awareness of citizens about the problems they cause when parking their bikes wrong. I noticed I never thought about the placement of my bike in relation to the rest of the pavement until I started this project. It makes you look with different eyes and makes you more aware of where to park your bike so enough space is left for others to pass through. 

\end{itemize}

\section{Final words}

It is proven that for elderly, who are more vulnerable, environmental attributes can be barriers to an active engagement in urban life. The quality of the immediate environment is a significant determinant of elders well-being, independence and quality of life. Developing ways to quantify their problems and add more data on their walkability quality could lead to more insight and hopefully to interventions on pavement level. We hope that with this research, a new methodology to quantify walkability is provided and makes the elderly pedestrian more visible in data and research. In the near future there will be more elderly who are willing to live independently and grow old in their own home. In order to facilitate this, we have to look forward and adapt the environment now that we are still young. 




 

