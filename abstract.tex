

\begin{abstract}

% Purpose
This research aims at grasping the critical walkability factors that elderly experience while walking outdoors with a rollator, and visualize them to raise more awareness and trigger possible action at decentralized governments. This may increase walkability walking activity of elderly people. Consequently it may support health, the capability to live independently and grow old in their own house and environment. All may contribute to less need for healthcare and so reducing costs for elder care.

% Methods
This is done, first, by finding the critical walkability factors for the outdoors urban environment through, literature study and interviews with elderly and the local government. Second, after selecting the most prominent factors, these will be mapped by using existing geo data and gis analysis to test its suitability and detail required. Third, a set of geo data is collected by measuring several rollator walks, where the rollator is mounted with GPS and an accelerometer to map movement and vibrations. Finally, the existing and collected data will be analysed and combined by using a method for change point detection.

% Results
The 3 most mentioned walkability hindrances named are, one, wrongly parked bikes and cars. Two, sloping pavements. And three irregular pavements. Surface hindrance was measured with an accelerometer on a rollator for 8 different surface types. 
Two routes with GPS and accelerometer were successfully covered and analysed. Yielding in data on, speed, distance, vibration movement. The slope map added height and slope to the route characteristics. 

% Conclusions
In conclusion, this pilot shows potential for using smart-phones with a simple accelerometer to detect obstacles and surface hindrance for rollator users. Though, in this study too many measurements failed no hard conclusions can be drawn and further, better research is needed too improve the amount of detail and accuracy in the measurements. 
The use of AHN data with a slope map did show good change point indication where a bump or ramp was taken. Though the accuracy of a smart-phone GPS, again, here gave not enough accuracy to draw hard conclusions from these measurements. 

\the\textwidth
\the\tabcolsep
\the\textwidth

\the\textlength

\end{abstract}
