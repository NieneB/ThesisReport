

\begin{abstract}

% Purpose
This research aims at grasping the critical walkability factors that elderly experience while walking outdoors with a rollator, and visualize them to raise more awareness and trigger possible action at decentralized governments. This may increase walking activity of elderly people. Consequently supporting health, the capability to live independently and grow old in their own house and environment. All may contribute to less need for healthcare and so reducing the costs for care, for the increasing ageing population.

% Methods
First, the critical walkability factors for elderly with a rollator, in the outdoors urban environment, are investigated through literature study. To have first hand user experiences, also interviews with elderly using a rollator were held and the local government of Amsterdam interviewed to get more insight in the policy designs. The top 3 most mentioned walkability hindrances named were, wrongly parked bikes and cars, sloping pavements and irregular pavements. GIS technologies can provide great potential for studying the environment characteristics that influence outdoor walkability quality. Using the GBKA a classification for mapping the pedestrian area is tested, the basis in a GIS system for pedestrian quality research. With the AHN2 the quality of these pavements is researched, deriving slope, to map the sloping pavement criteria. By collecting own data, with an measurement rollator, mounted with GPS and an accelerometer, the surface hindrance is mapped for different surface materials. Finally, a combination of the two datasets are analysed through change point detection. Mapping walking behaviour during a route (speed), surface hindrance (vertical acceleration) and comparing this against environmental characteristics (slope). The change point method shows an interesting concept for detecting obstacles during a walking route when linked to location. Several curb locations can be indicated where the walking behaviour (speed) decreases, the slope of the location changes and the accelerometer indicates peaks. Also some peaks in the vertical acceleration are located around imperfections in the pavement and are not accompanied by speed changes. 

% Conclusions
In conclusion, this pilot shows potential for using smart-phones with a simple accelerometer to detect obstacles and surface hindrance for rollator users. Though, in this study too many measurements failed and accuracy is low, no hard conclusions can be drawn. More test routes have to be conducted and better sensors with higher accuracy needed.
\end{abstract}
