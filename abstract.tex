

\begin{abstract}

% Purpose
This research aims at grasping the critical walkability factors that elderly experience while walking outdoors with a rollator and visualize these factors with GIS technologies.
% Methods
First, the critical walkability factors for elderly with a rollator in the outdoors urban environment, are investigated through literature study. Interviews with elderly using a rollator were held to get the user perspective and with the local government of Amsterdam to get insight in policy designs. The top 3 most mentioned walkability hindrances found in the literature and interviews were wrongly parked bikes and cars, sloping pavements and irregular pavements. GIS technologies can provide great potential for studying the environment characteristics that influence outdoor walkability quality. Using the GBKA a classification for mapping the pedestrian area is tested, which forms the basis in a GIS system for pedestrian quality research. With the AHN2 the quality of these pavements is researched, by deriving slope to map the critical boundary of the accepted 4\% slope. By collecting own data with a measurement rollator mounted with GPS and an accelerometer the surface hindrance is mapped for different surface materials. Finally, a combination of the two datasets is analysed through changepoint detection. We map walking behaviour during a route (speed), surface hindrance (vertical acceleration) and compare this against environmental characteristics (slope). Several curb locations can be indicated where the walking behaviour (speed) decreases, the slope of the location changes and the accelerometer indicates peaks. Also peaks in the vertical acceleration are located around imperfections in the pavement and are not accompanied by speed changes. 

% Conclusions
Many measurements failed and the GPS accuracy is low thus no hard conclusions can be drawn. More test routes should be conducted and better sensors with higher accuracy are needed. However, this pilot shows potential for a simple accelerometer to detect obstacles and surface hindrance for rollator users. Making these critical walkability factors easy to map and helps make visible where in the build environment improvement is needed. This could raise more awareness and trigger possible action at decentralized governments to improve walkability quality and so increase walking activity of elderly people. Consequently supporting health, the capability to live independently and grow old in their own house and environment. All contributes to less need for healthcare and so reducing the costs for care, for the increasing ageing population.

\begin{description}\item[Keywords:] Walkability, Rollator, Geo Information Systems, Smart Walker, Changepoint, Accelerometer
\end{description}
\end{abstract}
